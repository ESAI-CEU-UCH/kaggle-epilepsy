\documentclass[a4paper,english,twoside]{article}
\usepackage[T1]{fontenc}
\usepackage[utf8]{inputenc}
\usepackage{babel}
\usepackage{xspace}
\usepackage{amsmath}
\usepackage{epsfig}
\usepackage{graphicx}
\usepackage{fancyhdr}
\usepackage{calc}
\usepackage{anysize}
\usepackage{url}
\usepackage{hyperref}
\usepackage{libertine}
\usepackage[libertine]{newtxmath}

\marginsize{3cm}{3cm}{3cm}{3cm}

%\renewcommand{\familydefault}{\sfdefault}

%% %% Cabecera - Start %%
\fancyhf{}
\fancyhead[LO]{\small Seizure prediction system of ESAI-CEU-UCH team}
\fancyhead[RE]{\small F. Zamora-Martínez, F.J. Muñoz-Almaraz, P. Botella-Rocamora, J. Pardo}
\fancyhead[RO,LE]{}
\fancyfoot[C]{\bfseries \thepage}
\renewcommand{\headrulewidth}{0.5pt}  % headrule
\pagestyle{fancy}
%% %% Cabecera - End %%

\newcommand{\etc}{etc.\@\xspace}
% \setlength{\extrarowheight}{-5em}

\newcommand{\argmax}{\mathop{\mbox{argmax}}}

% \name{}
\author{Francisco Zamora-Martínez, Francisco J. Muñoz-Almaraz,\\
  Paloma Botella-Rocamora, Juan Pardo}
\date{November 2014}
% \address{}
\title{\ Seizure prediction using FFT, eigen values of correlation matrix,
  artificial neural networks, k-nearest-neighbors and Bayesian model combination}

\begin{document}
\maketitle

\begin{itemize}
\item[Location:] Universidad CEU Cardenal Herrera, Alfara del Patriarca,
  Valencia, Spain. Embedded Systems and Artificial Intelligence (ESAI) research
  group.
\item[Email:] \{francisco.zamora, malmaraz, pbotella, juaparal\}@uch.ceu.es
\item[Competition:] American Epilepsy Society Seizure Prediction Challenge.
\end{itemize}

\section{Summary}\label{summary}

This report presents the solution proposed by ESAI-CEU-UCH team, composed by
lecturers from Universidad CEU Cardenal Herrera, at Kaggle American Epilepsy
Society Seizure Prediction Challenge~\cite{kaggle}. The proposed solution
positioned us at the \textbf{4th} place at Kaggle leaderboard. Different kind of
input features (different preprocessing pipelines) and different statistical
models are being proposed. This diversity was motivated to allow model
combination to improve the result.

It is important to note that none of the proposed systems use the test set
for any calibration or related purpose. The competition rules allow to do a
model calibration using test set, but doing it will reduce the
reproducibility of the results in a real world implementation.

\section{Feature selection and
  extraction}\label{feature-selection-and-extraction}

Several preprocessing pipelines have been performed. The most relevant
preprocessing technique according to final performance were Fast Fourier
Transform (FFT) plus Principal Components Analysis (PCA) procedure. This
different pipelines combine methods implemented in R~\cite{Rproject} and others
in APRIL-ANN toolkit~\cite{aprilann}.

\subsection{FFT features plus PCA/ICA}\label{fft-features-plus-pcaica}

This is the most important kind of proposed features, obtaining the best
standalone result. This process is similar to~\cite{2014:howbert:plosone}, and
has been implemented into APRIL-ANN toolkit. For each input file it follows
the next steps:

\begin{enumerate}
\item
  Extract windows of 60 seconds size with 50\% overlapping for every
  channel in the data, increasing the number of samples in the data set.
  We obtained 19 windows for every input file. Every sample is filtered
  by using a Hamming window.
\item
  For every window, real FFT is computed using an algorithm which needs power of
  two window sizes. The 60 seconds input window has 24\,000 samples for Dogs and
  300\,000 samples for Patients, leading into 16\,384 FFT bins for Dogs and
  262\,144 FFT for Patients.
\item A filter bank is computed for bands: delta (0.1Hz --- 4Hz), theta (4Hz ---
  8Hz), alpha (8Hz --- 12Hz), beta (12Hz --- 30Hz), low-gamma (30Hz --- 70Hz),
  high-gamma (70Hz --- 180Hz). The mean in the corresponding frequency bands has
  been computed.
\item
  The output of the filter bank is compressed by computing the logarithm
  of the values.
\end{enumerate}

For every file, this FFT preprocess computes a matrix with 19 rows (windows of
60 seconds with 50\% overlapping) and $6 \cdot C$ columns being $C$ the number of
channels. This transformation has been applied to all the available data
(training and test sets). It has been implemented in APRIL-ANN toolkit by using
functions \verb+matlab.read+ and \verb+matrix.ext.real_fftwh+.

A PCA transformation has been computed over the concatenation by rows of all
training data matrices (ignoring test data). Data have been centered and scaled
before PCA. After computing centers, scales and PCA matrix transformation,
have been applied to all the available data (training and test). In a similar
fashion to PCA, Independent Component Analysis (ICA) has been performed. The
advantage of PCA/ICA transformation is their ability to remove linear
correlations between features.\footnote{ICA transformation needs a random number
  generator and the seed was random during competition, so it is not possible to
  reproduce exactly the competition result.} PCA and ICA transformations have
been computed using \verb+prcomp+ and \verb+fastICA+ of R. The
transformations have been applied to all the data using APRIL-ANN.

\subsection{Eigen values of pairwise channels correlation
  matrix}\label{eigen-values-of-pairwise-channels-correlation-matrix}

This preprocessing was based on previous competition winner
report~\cite{michaelhills} presented by Michael Hills.

\subsubsection{Windowed correlation
  matrix}\label{windowed-correlation-matrix}

Correlations between the channels with windows covering 60s were calculated and
the eigenvalues of the correlation matrix were taken as features for training
with K-Nearest-Neighbors (KNN) and Artificial Neural Network (ANN) models (see
Section~\ref{modelling-techniques-and-training}).  Every 10 minutes segment
provided in the competition data set generates 19 signal subintervals whose
lapse is 60s overlapping 30s with the previous and the posterior subintervals.
Each subinterval corresponds to a data set that consists in a matrix whose size
is the number of channels times the product of the sampling frequency and 60.
For every subinterval (window covering 60s) the eigenvalues of its correlation
matrix were calculated and stored in a new file. With the former procedure, the
number of covariables is $19 \cdot C$ being $C$ the number of channels in the
subject.  Therefore, the number of features is different depending on the
subject. These features jointly with FFT (with and without PCA/ICA
transformation) were used as inputs, and denoted as FFT+CORW,
PCA+CORW, ICA+CORW.\footnote{PCA/ICA transformation is only applied to FFT
  features, not to CORW features.} R has been used to implement this step using
functions \verb+cor+ and \verb+eig+.

Correlations were also calculated with the FFT transformation of every
sequence described in section~\ref{fft-features-plus-pcaica},
but the results were not stable enough.

\subsubsection{Global correlation
  matrix}\label{global-correlation-matrix}

A whole segment file provided in the competition was analysed to find
correlations between channels.  The set of features obtained with this analysis
was used with KNN models for training.  Not only was considered the original
signal in the sequence, but also a differentiated signal consisting in the
difference of two consecutive values of the original signal. The original signal
is a matrix whose size is the number of channels times the product of the
frequency sampling and 600 (60 seconds times 10 minutes). The functions
\verb@eigen@ and \verb@cor@ in R are applied to this matrix and the output are
the eigenvalues of the correlation matrix. The number of eigenvalues is equal to
the number of channels for each subject.

For each channel, a transformed signal is obtained with the difference between
two consecutive values of the signal.  The correlations of these differentiated
signals form new features that added to the correlation of the original signals
increase the predictive performance with KNN models. The differentiated signals
of a segment is a matrix whose size is the number of channels times the product
of the frequency sampling and 599 (60 seconds times 10 minutes minus one second
needed to compute differences). Applying \verb@eigen@ and \verb@cor@ functions,
the eigenvalues of the correlation matrix of these differentiated signals were
calculated.

With these procedures, the number of features doubles the number of
channels, and they were stored in a new file. This features will be denoted as CORG
from now on.

The main flaw observed is that for human patients the differentiated signal does
not provide useful information due to patient data have a higher sampling
frequency. We suppose that the difference of two values in a human signal,
separated by a lapse of time similar to the dog signals (1/400s), could improve
the quality of these features.  Nevertheless, being a minor feature, in the
final track of the competition we payed more attention to the most prominent
preprocessing.

\subsection{Global statistical
  features}\label{global-statistical-features}

Exploratory analysis of original data showed different behavior in the signal
variability of both classes, pre-ictal and inter-ictal. Standard deviation were
calculated for every data set in every channel.  Following the study of
discriminant statistical features, a fourier basis representation (with 15
elements) was calculated for every data set. Exploratory analysis on their
coefficients showed different behavior between pre-ictal and inter-ictal
series. Pre-ictal signals coefficients displayed lower coefficients (closer to
0) while inter-ictal signals coefficients presented greater disparity between
them. As a summary measure of this behavior, standard deviation of these
coefficients was calculated for each data serie, once per channel, and the mean
standard deviation was calculated for all channels to reduce the dimensionality
and avoid redundant information. This feature will be denoted as COVRED from now
on. It has been implemented in R by using the functions \verb+sd+, \verb+mean+
and \verb+fdata2fd+,

\section{Modelling techniques and
  training}\label{modelling-techniques-and-training}

\subsection{Cross-validation
  algorithm}\label{cross-validation-algorithm}

A cross-validation approach has been followed to compute Area Under Curve (AUC)
estimate using the training data. All models have been trained independently, but
using the same hyper-parameters and configurations.

The number of folds per subject is the maximum number of full
sequences\footnote{A full sequence are those were all sequence numbers are
  available. Note that not all the given sequences in training data are full.}
in pre-ictal data. Folds have been build in such a way that all segments in the
same sequence belong to the same fold. Both, pre-ictal and inter-ictal data have
been splitted into the same number of folds, and following the same sequence
constraint.  For KNNs, cross-validation allows to compute an estimate of the
AUC, but the final model contains all the available training data.  For ANNs,
one model has been trained for each possible partition, and the test result is
the mean of all trained model outputs.

This cross-validation algorithm has been implemented carefully for APRIL-ANN in
file \verb+common.lua+, function \verb+common.train_with_crossvalidation+,
whose most important arguments are the trainining and classify functions.

\subsection{Models}\label{models}

Different models have been tested in APRIL-ANN toolkit. Logistic regression as
initial baseline following a simple linear regression approach, which has not
been introduced into the final system due to convergence problems.  In order to
exploit non-linear relations in data, KNNs and dropout ANNs with different
number of layers and Rectified Linear Units~\cite{2011:glorot:aistats} (ReLUs)
as neurons have been trained. And finally, an ensemble of several models has
been performed.

\subsubsection{General aspects}

Two different model specifications are possible, depending on the kind of input
features: global features model and windowed features model. The case of global
features model is the most simple, correspond to models whose inputs are CORG
and COVRED features (see Sections~\ref{global-correlation-matrix}
and~\ref{global-statistical-features}). The model receives as input all the
features extracted from one segment file, and computes how likely this file is a
pre-ictal sample. In case of windowed features model, the model receives a file
segment splitted following a sliding window over temporal axis, and computes how
likely every window is a precital sample. Once all window probabilities have
been computed, pre-ictal probability of the file segment is computed as the
complementary of the geometric mean of inter-ictal probabilities, using
APRIL-ANN code \verb+( 1 -stats.gmean(1 - p) )+, following this equation:

\begin{equation}
p(\text{pre-ictal} | x) = 1 - \sqrt[n]{\prod_{t=1}^n (1 - p_t)}
\label{eq:gmean}
\end{equation}

\noindent where $x$ is a file segment, $n$ is the number of time windows in $x$
and $p_t$ is the pre-ictal probability of window $t$. Different aggregation
methods, as arithmetic mean, maximum or harmonic mean, were tested achieving
similar or worst results.

\subsubsection{K-Nearest Neighbors}\label{k-nearest-neighbors}

KNN models were chosen after convergence problems with logistic regression
models. KNNs are non-parametric models and don't need to be trained because
training data are the model itself. The KNN model has a hyper-parameter that is
the number of neighbors, denoted by $K$. This hyper-parameter has been set to
$K=40$ after trying different values and looking into cross-validation and
public AUC scores. The $K$ neighbors given by the KNN were transformed into
probabilities following the implementation given in APRIL-ANN toolkit (function
\verb+knn.kdtree.posteriorKNN+), which is based on~\cite{2005:nips:hinton:NCA},
and basically computes a posterior probability by normalizing the exponential of
the negative distances, following this equation:

\begin{equation}
p(\text{pre-ictal} | s) =
\displaystyle{\frac{\displaystyle{\sum_{ s^\prime \in \text{pre-ictal}(\mathcal{K}) } exp( -||s - s^\prime||^2_2 ) }}
{\displaystyle{\sum_{s^\prime \in \mathcal{K}} exp( -||s - s^\prime||^2_2 ) }}}
\end{equation}

\noindent where $s$ is an input sample\footnote{It can be a global feature
  vector computed over the whole segment file, or a temporal window taken from
  the file.}, $\mathcal{K}$ is the set of $K$-neighbors given by the KNN,
$\text{pre-ictal}(\mathcal{K})$ is a subset of $\mathcal{K}$ containing only
samples belonging to pre-ictal class. The adequate value of $K$ and the
computation of probabilities using distances reduce the impact of overfitting in
the KNN model.

\subsubsection{Artificial Neural
  Networks}\label{artificial-neural-networks}

ANN models with different hyper-parameters were tested, but in a non exhaustive
way because of the large gap between cross-validation AUC and public test
AUC. Overfitting effect was reduced by introducing a 50\%
dropout~\cite{2012:arxiv:hinton:dropout} in all hidden layers and a Gaussian
additive noise with mean $0$ and variance $0.04$ in the inputs. All proposed
ANNs have ReLU~\cite{2011:glorot:aistats} as activation functions\footnote{ReLU
  follows the expression $r(x) = max(0,x)$.} in all hidden layers, and logistic
activation function at the output layer. Training was performed by stochastic
gradient descent with mini-batches (bunch size) of 128 samples and minimizing
the cross-entropy loss function. Validation loss was also cross-entropy but
after aggregating all the probabilities computed for each segment file following
the Equation~(\ref{eq:gmean}). The input of the ANN was extended taking into
account three windows (the current, the previous one and the following),
allowing to capture temporal patterns in data. The learning rate was set to
$\eta_0=0.2$ with a decay parameter of $\epsilon=0.001$, momentum was set to
$\gamma=0.1$ and no other regularization term has been used. ANN weights were
initialized sampling from a uniform distribution in range $[-3/\sqrt{\text{fan-in} +
  \text{fan-out}},3/\sqrt{\text{fan-in} + \text{fan-out}}]$. In APRIL-ANN one weight
$w$ at iteration $t+1$ is computed following:

\begin{equation}
w^{(t+1)} = w^{(t)} - \frac{\eta_0}{1 + \epsilon t} \cdot \frac{\partial L}{\partial w^{(t)}} + \gamma ( w^{(t)} - w^{(t-1)} )
\end{equation}

\noindent being $L$ the cross-entropy loss of the mini-batch.

Different hyper-parameters, including L1 and L2 regularization were tested without
good results. Again, it is important to take into account that all these
hyper-parameters were selected in a non-exhaustive way.

\subsubsection{Ensemble of models}\label{ensemble-of-models}

All the proposed models have been trained independently, computing
cross-validation probabilities for every training file, and test output
submission file, and the final system was a combination of these outputs.
Initially, a simple uniform linear combination of probability outputs in
submission files has been tested. In order to improve such result, a Bayesian
Model Combination (BMC) algorithm~\cite{2011:monteith:ijcnn} has been
implemented in Lua for APRIL-ANN~\cite{aprilann} toolkit. BMC algorithm uses
cross-validation output probabilities to optimize the likelihood of the
combination, and uses these weights to combine output probabilities in
submission files.

\subsection{Results}\label{results}

The best submitted model is the BMC ensemble of the described models
above. It achieved a \textbf{0.7935} AUC in private test.

\section{Dependencies and solution recipe}\label{dependencies}

This system uses the following open source software:

\begin{itemize}
\item APRIL-ANN toolkit v0.4.0~\cite{aprilann}. It is a toolkit for pattern
  recognition with Lua scripting and a C/C++ core.  Because this tool is very
  new, the installation and configuration has been written in the pipeline.
\item R project v3.0.2~\cite{Rproject}. For statistical
  computing, a wide spread tool in Kaggle competitions. Packages
  R.matlab, MASS, fda.usc, fastICA, stringr and plyr are necessary to
  run the system.
\item GNU BASH v4.3.11~\cite{bash} with standard command line tools.
\end{itemize}

The system is prepared to run in a Linux system with Ubuntu 14.04 LTS, but it
can be run in other Debian based distributions, but not tested.

It is possible to run the system (training and test) by executing:

\begin{verbatim}
$ ./train.sh
\end{verbatim}

\noindent And it is possible to run only test by using:

\begin{verbatim}
$ ./test.sh
\end{verbatim}

See the README file at GitHub
repository\footnote{\url{https://github.com/ESAI-CEU-UCH/kaggle-epilepsy}} for
deeper information about the solution recipe.

\section{Code description}\label{code-description}

The code is a bunch of different scripts for APRIL-ANN and R, plus bash-scripts
for the recipe automation. The folder \verb+scripts/+ contains the different
programs, which have been organized depending in their purpose.

The following scripts are for configuration and common functions:

\begin{itemize}
\item \verb+scripts/configure.sh+: is a bash-script which downloads APRIL-ANN,
  install dependencies, and compiles it using ATLAS library by default.
\item \verb+scripts/configure.R+: is an R script which install the needed
  R packages at the user home folder.
\item \verb+scripts/cmd.lua+: contains Lua code for APRIL-ANN with command
  line options of the training scripts.
\item \verb+scripts/common.lua+: contains Lua code for APRIL-ANN with several
  functions which were shared between different preprocessing and training
  programs.
\end{itemize}

The following ones are for preprocessing:

\begin{itemize}
\item \verb+scripts/PREPROCESS/compute_fft.lua+: is a Lua script for APRIL-ANN
  which computes the FFT preprocess, before PCA/ICA transformation, indicated at
  Section~\ref{fft-features-plus-pcaica}.
\item \verb+scripts/PREPROCESS/compute_pca.R+: is an R script which computes
  PCA transformation matrices: rotation, center and scale matrices.
\item \verb+scripts/PREPROCESS/compute_ica.R+: is an R script which computes ICA
  transformation matrices: rotation, center and scale matrices.
\item \verb+scripts/PREPROCESS/apply_pca.lua+: is a Lua script for APRIL-ANN
  which takes the PCA transformation matrices and computes PCA rotation of all
  data.
\item \verb+scripts/PREPROCESS/apply_ica.lua+: is a Lua script for APRIL-ANN
  which takes the ICA transformation matrices and computes ICA of all data.
\item \verb+scripts/PREPROCESS/correlation_60s_30s.R+: is a R script which
  computes windowed correlation matrix as described at
  Section~\ref{windowed-correlation-matrix}.
\item \verb+scripts/PREPROCESS/compute_corg.R+: is a R script which computes
  global correlation matrix as described at
  Section~\ref{global-correlation-matrix}.
\item \verb+scripts/PREPROCESS/compute_covarred.R+: is a R script which computes
  global statistical features as described at
  Section~\ref{global-statistical-features}.
\end{itemize}

\noindent Note that this preprocessing scripts produces intermediary output
results in disk storage.

The following are for training+testing:

\begin{itemize}
\item \verb+scripts/MODELS/train_one_subject_knn.lua+: is a Lua script for
  APRIL-ANN which computes cross-validation results for KNN model and produces
  the test outputs. This script has a lot of configuration options, the most
  promising were selected during competition and fixed in the solution at
  \verb+scripts/MODELS/confs/knn*.lua+.
\item \verb+scripts/MODELS/train_one_subject_mlp.lua+: is a Lua script for
  APRIL-ANN which trains a MLP following the cross-validation scheme and
  produces its outputs. This script has a lot of configuration options, the most
  promising were selected during competition and fixed in the solution at
  \verb+scripts/MODELS/confs/ann*.lua+.
\item \verb+scripts/MODELS/train_all_subjects_wrapper.lua+: is a Lua script for
  APRIL-ANN which takes one of the two above scripts and trains the model for
  all the available subjects, producing the submission for Kaggle.
\end{itemize}

The following are for models ensemble:

\begin{itemize}
\item \verb+scripts/ENSEMBLE/naive_ensemble.lua+: is a Lua script for
  APRIL-ANN which implements an uniform linear combination of several test
  outputs.
\item \verb+scripts/ENSEMBLE/bmc_ensemble.lua+: is a Lua script for APRIL-ANN
  which implements BMC ensemble for models trained following the
  cross-validation scheme, and produces the corresponding test output.
\end{itemize}

The following are only for testing:

\begin{itemize}
\item \verb+scripts/MODELS/test_one_subject_knn.lua+: is a Lua script for
  APRIL-ANN which computes test results for a KNN model and one subject.
\item \verb+scripts/MODELS/test_one_subject_mlp.lua+: is a Lua script for
  APRIL-ANN which computes test results for a MLP model and one subject.
\end{itemize}

\section{Additional comments and
  observations}\label{additional-comments-and-observations}

Among the indicated as submitted best system, different models and
features combinations have been tested. Table~\ref{tab:val} summarizes
the cross-validation AUC and public test AUC for the most important
combinations. In Table~\ref{tab:val} first column contains the model, being one
of the following:

\begin{itemize}
\item
  LR: Logistic regression.
\item
  KNN: K-Nearest-Neighbors with $K=40$.
\item ANN: Artificial Neural Networks with 50\% dropout and
  ReLU~\cite{2011:glorot:aistats} neurons, 128 units in every hidden layer, and
  a logistic activation function at the output.  ANN2 for two hidden layers,
  ANN3 for three hidden layers, and so on. ANN2p for two hidden layers with 64
  units in every layer,
\item
  UNIFORM or BMC: ensemble of the most promising systems, indicated in
  bold face. UNIFORM is a linear combination with uniformly distributed
  weights; BMC is linear combination where weights are estimated
  following BMC optimizing cross-validation likelihood.
\end{itemize}

The second column indicates which features were used as input of the
model, being one of the following list:

\begin{itemize}
\item
  FFT: the output of the proposed filter bank plus logarithm
  compression.
\item
  FFT+CORW: the previous one plus windowed eigen values of correlation
  matrix.
\item
  PCA+CORW: the PCA transformation of FFT features plus windowed eigen
  values of correlation matrix.
\item
  ICA+CORW: idem as previos one, but using ICA instead of PCA.
\item
  CORG: correlation using 10 minutes window.
\item
  COVRED: different statistical measures over the original signal.
\item
  ENSEMBLE: output probabilities of the bold faced systems.
\end{itemize}

\begin{table}
  \centering
  \begin{tabular}{|l|l|cc|}
    \hline
    Model & Features & CV AUC & Pub. AUC\\
    \hline
    \hline
    LR & FFT & 0.9337 & 0.6784\\
    \hline
    KNN & FFT & 0.8008 & 0.6759\\
    KNN & FFT+CORW & 0.7994 & 0.7040\\
    \textbf{KNN} & \textbf{PCA+CORW} & \textbf{0.8104} & \textbf{0.7288}\\
    \textbf{KNN} & \textbf{ICA+CORW} & \textbf{0.8103} & \textbf{0.6840}\\
    \hline
    ANN2 & FFT+CORW & 0.9072 & 0.7489\\
    ANN2 & PCA+CORW & 0.9082 & 0.7815\\
    \textbf{ANN2p} & \textbf{PCA+CORW} & \textbf{0.9175} & \textbf{0.7895}\\
    \textbf{ANN2} & \textbf{ICA+CORW} & \textbf{0.9104} & \textbf{0.7772}\\
    ANN3 & PCA+CORW & 0.9188 & 0.7690\\
    ANN4 & PCA+CORW & 0.9268 & 0.7772\\
    \textbf{ANN5} & \textbf{PCA+CORW} & \textbf{0.9283} & \textbf{0.7937}\\
    ANN6 & PCA+CORW & 0.9291 & 0.7722\\
    \hline
    \textbf{KNN} & \textbf{CORG} & \textbf{0.7097} & \textbf{0.6552}\\
    \textbf{KNN} & \textbf{COVRED} & \textbf{0.6900} & \textbf{0.6901}\\
    \hline
    UNIFORM & ENSEMBLE & --- & 0.8048\\
    BMC & ENSEMBLE & 0.9271 & 0.8249\\
    \hline
  \end{tabular}
  \caption{Cross-validation and public AUC results for the most important
    systems tested during the competition.\label{tab:val}}
\end{table}            

Table~\ref{tab:val} shows that logistic regression model was the worst in
terms of generalization ability. The application of PCA allows to
improve the public AUC results by \textbf{0.025} for KNN model and
\textbf{0.033} for ANN model. ICA obtains improvements similar to PCA.
The advantage of deep ANNs (with more than two hidden layers) is not
clear, but the best public AUC was obtained by an ANN with five hidden
layers, improving by \textbf{0.004} the public AUC of the best ANN2.
Global features as CORG and COVRED show strong correlation between
cross-validation AUC and public test AUC. Finally, the use of BMC
ensemble method allow to improve the public AUC in \textbf{0.031} points
compared to the ANN5 model.

Finally, the private AUC for the two selected models is shown at
Table~\ref{tab:private}. Note that this table shows private AUC for competition
system, and private AUC for post-competition system using MKL or ATLAS
library. As it is indicated in the README at GitHub repository, the competition
system couldn't be reproduce in an exact way due to a non fixed seed number.
Now, the code has this number fixed, and its AUC is shown in the table. In
addition to this problem, our system uses Intel MKL library for the best performance
in experimentation. However, the code is configured by default to compile using
ATLAS library, which is open source and standard in Linux systems. The
post-competition private AUC for MKL and ATLAS options are also shown in the
table for comparison purposes.\footnote{Note that the system hyper-parameters
  have been optimized using MKL compilation flag, so it is expected worst
  results compiling with ATLAS library.}

\begin{table}
  \centering
  \begin{tabular}{|l|l|c||c|c|}
    \hline
          &          & Priv. AUC & \multicolumn{2}{|c|}{Post. priv. AUC}\\
    Model & FEATURES & MKL & \multicolumn{1}{|c}{MKL} & \multicolumn{1}{c|}{ATLAS} \\
    \hline
    \hline
    ANN5 & PCA+CORW & 0.7644          & 0.7644 & 0.7405\\
    BMC & ENSEMBLE & \textbf{0.7935}  & 0.7937 & 0.7910\\
    \hline
  \end{tabular}
  \caption{Private AUC for the two selected submissions. Post. priv. AUC refers
    to AUC results in private test after the competition.\label{tab:private}}
\end{table}

The private AUC of the BMC ensemble is the second best if we compare it
to all our submissions. However, the difference with the best private
AUC is not significant, the best one has a private AUC of
\textbf{0.7958}.

The experimentation has been performed in a cluster of \textbf{three}
computers with same hardware configuration:

\begin{itemize}
\item
  Dell PowerEdge 210 II server rack.
\item
  Intel Xeon E3-1220 v2 at 3.10GHz with 16GB RAM (4 cores).
\item
  2.6TB of NFS storage.
\end{itemize}

\section{Simple features and methods}\label{simple-features-and-methods}

The best single features plus model was ANN5, an artificial neural
network with 5 hidden layers using ReLUs~\cite{2011:glorot:aistats}
as activation function, with 128 neurons in every layer and logistic
activation function as output. The input features were FFT+CORW as
described above. This system achieves a \textbf{0.7644} AUC in the
private test.

\section*{Post-competition software revision: v1.0}

A version 1.0 of the software has been prepared, just solving bugs which have
been introduced during competition. This new version removes from the ensemble
combination the ANN models with PCA features which lead to a worst public AUC
score. So, compiled with Intel MKL option, the BMC ensemble of ANN2 with ICA
features and KNNs with ICA, PCA, CORG and COVRED features achieves a public AUC
of $\mathbf{0.8222}$ and a private AUC of $\mathbf{0.7947}$. The models
selection for the ensemble has been performed maximizing the public AUC.

\bibliography{refs}
\bibliographystyle{plain}

\end{document}
