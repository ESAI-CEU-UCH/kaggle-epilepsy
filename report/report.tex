\documentclass[a4paper,english,twoside]{article}
\usepackage[T1]{fontenc}
\usepackage[utf8]{inputenc}
\usepackage{babel}
\usepackage{xspace}
\usepackage{amsmath}
\usepackage{epsfig}
\usepackage{graphicx}
\usepackage{fancyhdr}
\usepackage{calc}
\usepackage{anysize}
\usepackage{url}
\marginsize{3.5cm}{3cm}{3.5cm}{3.5cm}

%% %% Cabecera - Start %%
\fancyhf{}
\fancyhead[LO,RE]{\small \emph{TITLE}}
\fancyhead[RO,LE]{}
\fancyfoot[C]{\bfseries \thepage}
\renewcommand{\headrulewidth}{0.5pt}  % headrule
\pagestyle{fancy}
%% %% Cabecera - End %%

\newcommand{\etc}{etc.\@\xspace}
% \setlength{\extrarowheight}{-5em}

\newcommand{\argmax}{\mathop{\mbox{argmax}}}

% \name{}
\author{Francisco Zamora Martínez, Francisco J. Mu\~noz--Almaraz,\\
  Paloma Botella--Rocamora, Juan Pardo}
\date{November 2014}
% \address{}
\title{Seizure prediction using FFT, eigen values of correlation matrix,
  artificial neural networks, k-nearest-neighbors and Bayesian model combination}
% \school{Departamento de Sistemas Inform'aticos y Computaci'on \\
% Universidad Polit'ecnica de Valencia, Valencia (Spain)}
% \email{fzamora@dsic.upv.es}

\begin{document}
\maketitle

\begin{itemize}
\item[Location:] Universidad CEU Cardenal Herrera, Alfara del Patriarca,
  Valencia, Spain. Embedded Systems and Artificial Intelligence (ESAI) research
  group.
\item[Email:] \{francisco.zamora, malmaraz, pbotella, juaparal\}@uch.ceu.es
\item[Competition:] American Epilepsy Society Seizure Prediction Challenge.
\end{itemize}

\section{Summary}\label{summary}

This work presents the system proposed by Universidad CEU Cardenal
Herrera (ESAI-CEU-UCH) at Kaggle American Epilepsy Society Seizure
Prediction Challenge. The proposed system was positioned as \textbf{4th}
at Kaggle American Epilepsy Society Seizure Prediction Challenge~\cite{kaggle}.

Different kind of input features (different preprocessing pipelines) and
different statistical models are being proposed. This diversity was motivated to
improve model combination result.

It is important to note that none of the proposed systems use test set
for any calibration or related purpose. The competition rules allow to do this
model calibration using test set, but doing it will reduce the
reproducibility of the results in a real world implementation.

\section{Feature selection and
  extraction}\label{feature-selection-and-extraction}

Several preprocessing pipelines have been performed. The  most relevant
preprocessing technique according to final performance is  FFT plus
PCA procedure.

\subsection{FFT features plus PCA/ICA}\label{fft-features-plus-pcaica}

This is the most important kind of proposed features, obtaining the best
standalone result. This process follows the following steps for each input
file~\cite{2014:howbert:plosone}:

\begin{enumerate}
\item
  Extract windows of 60 seconds size with 50\% overlapping for every
  channel in the data, increasing the number of samples in the data set.
  We obtained 19 windows for every input file. Every sample is filtered
  by using Hamming windows.
\item
  For every window, real FFT is computed using an algorithm which needs
  power of two window sizes, so, 60s input windows are of 24\,000 for Dogs
  and 300\,000 for Patients, leading into 16384 FFT bins for Dogs and
  262144 FFT for Patients.
\item A filter bank is computed for bands: delta (0.1Hz --- 4Hz), theta (4Hz ---
  8Hz), alpha (8Hz --- 12Hz), beta (12Hz --- 30Hz), low-gamma (30Hz --- 70Hz),
  high-gamma (70Hz --- 180Hz). The mean in the corresponding frequency bands has
  been computed.
\item
  The output of the filter bank is compressed by computing the logarithm
  of the values.
\end{enumerate}

For every file, it is computed a matrix with 19 rows (windows of 60s with 50\%
overlapping) and $6C$ columns being $C$ the number of channels. This FFT
transformation has been applied for all the available data, training and test.

A PCA transformation has been computed over the concatenation by rows of
all training data matrices, leaving test data hidden for this step. Data
has been centered and scaled before PCA. After computing centers, scales
and PCA matrix transformation, it has been applied to all the available
data, training and test. In a similar fashion as for PCA, Indenpendent
Component Analysis (ICA) has been performed. The advantage of PCA/ICA
transformation is their ability to remove linear correlations between
pairs of features.

\subsection{Eigen values of pairwise channels correlation
  matrix}\label{eigen-values-of-pairwise-channels-correlation-matrix}

This preprocessing was based in previous competition system of Michael Hills~\cite{michaelhills}.

\subsubsection{Windowed correlation
  matrix}\label{windowed-correlation-matrix}

\subsubsection{Global correlation
  matrix}\label{global-correlation-matrix}

\subsection{Global statistical
  features}\label{global-statistical-features}

\section{Modelling techniques and
  training}\label{modelling-techniques-and-training}

\subsection{Cross-validation
  algorithm}\label{cross-validation-algorithm}

A cross-validation approach has been followed to compute AUC estimate using
training data. All models have been trained independently, but using the same
hyper-parameters and configurations.

The number of folds per subject is the maximum number of full
sequences\footnote{A full sequence are those where all sequence numbers are
  available, not all the available sequences are full.} in preictal data. Folds
are build in such way that all segments in the same sequence are together in
the same fold. Both, preictal and interictal data has been splitted into the
same number of folds, and following the same sequence constraint.

For KNNs, cross-validation allows to compute an estimate of the AUC, but the
final model contains all the available training data.

For ANNs, one model has been trained for each possible partition, and the test
result is the mean of the output of every model.

\subsection{Models}\label{models}

Different models has been tested. Logistic regression as initial
baseline following a simple linear regression approach, which was not
introduced in the final system. KNNs for a more sophisticated model. In
order to exploit non-linear relations in data, dropout ANNs with
different number of layers and ReLU as neurons. And finally, an ensemble
of several models has been performed.

\subsubsection{General aspects}

Two different model specifications are possible, depending in the kind of input
features: global features model and windowed features model. The case of global
features model is the most simple, correspond to models which input is CORG and
COVRED features. This model receives as input all the features extracted from
one segment file, and computes how likely this file is a preictal sample. In
case of windowed features model, the model receives a file segment splitted
following a sliding window over temporal axis, and computes how likely every
window is a precital sample. Once all windows probabilities have been computed,
preictal probability of the file segment is computed as the complementary of a
geometric mean:

\begin{equation}
p(c=\text{preictal} | x) = 1 - \sqrt[n]{\prod_{t=1}^n (1 - p_t)}
\end{equation}

\noindent where $x$ is a file segment, $p_t$ is the probability of window $t$
and $n$ is the number of windows.

\subsubsection{K-Nearest Neighbors}\label{k-nearest-neighbors}

\subsubsection{Artificial Neural
  Networks}\label{artificial-neural-networks}

\subsubsection{Ensemble of models}\label{ensemble-of-models}

All the proposed models have been trained independently, computing
cross-validation probabilities for every training file, and test output
submission file.

Initially, a simple uniform linear combination of probability outputs in
submission files has been tested. In order to improve this result, a Bayesian
Model Combination (BMC) algorithm~\cite{2011:monteith:ijcnn} has been
implemented in Lua for APRIL-ANN~\cite{aprilann} toolkit. BMC algorithm uses
cross-validation output probabilities to optimize the combination weights, and
uses these weights to combine output probabilities in submission files.

\subsection{Results}\label{results}

The best submitted model is the combination of the described models
above. They achieve a \textbf{0.79335} AUC in private test.

\section{Code description}\label{code-description}

\section{Dependencies}\label{dependencies}

\subsection{Software}\label{software}

This system uses the following open source software:

\begin{itemize}
\item APRIL-ANN toolkit v0.4.0~\cite{aprilann}.
  It is a toolkit for pattern recognition with Lua and C/C++ core.
  Because this tool is very new, the installation and configuration has
  been written in the pipeline.
\item R project v3.0.2~\cite{Rproject}. For statistical
  computing, a wide spread tool in Kaggle competitions. Packages
  R.matlab, MASS, fda.usc, fastICA, stringr and plyr are necessary to
  run the system.
\item GNU BASH v4.3.11~\cite{bash} with find, sort and tr command line tools.
\end{itemize}

The system is prepared to run in a Linux system with Ubuntu 14.04 LTS, but it
can be run in other Debian based distributions, but not tested.

\subsection{Hardware}\label{hardware}

The minimum requirements for the correct execution of this software are:

\begin{itemize}
\item
  6GB of RAM.
\item
  1.5GB of disk space.
\end{itemize}

\section{How to generate the
  solution}\label{how-to-generate-the-solution}

\section{Additional comments and
  observations}\label{additional-comments-and-observations}

Among the indicated as submitted best system, different models and
features combinations has been tested. Table~\ref{tab:val} summarizes
the cross-validation AUC and public test AUC for the most important
combinations. In Table~\ref{tab:val} first column contains the model, being one
of the following:

\begin{itemize}
\item
  LR: Logistic regression.
\item
  KNN: K-Nearest-Neighbors with $K=40$.
\item ANN: Artificial Neural Networks with 50\% dropout and
  ReLU~\cite{2011:glorot:aistats} neurons, 128 units in every hidden layer, and
  a logistic activation function at the output.  ANN2 for two hidden layers,
  ANN2p for two hidden layers with 64 units in every layer, ANN3 for three
  hidden layers, and so on.
\item
  UNIFORM or BMC: ensemble of the most promising systems, indicated in
  bold face. UNIFORM is a linear combination with uniformly distributed
  weights; BMC is linear combination where weights are estimated
  following BMC optimizing cross-validation likelihood.
\end{itemize}

The second column indicates which features were used as input of the
model, being one of the following list:

\begin{itemize}
\item
  FFT: the output of the proposed filter bank plus logarithm
  compression.
\item
  FFT+CORW: the previous one plus windowed eigen values of correlation
  matrix.
\item
  PCA+CORW: the PCA transformation of FFT features plus windowed eigeb
  values of correlation matrix.
\item
  ICA+CORW: idem as previos one, but using ICA instead of PCA.
\item
  CORG: correlation using 10 minutes window.
\item
  COVRED: different statistical measures over the original signal.
\item
  ENSEMBLE: output probabilities of the bold faced systems.
\end{itemize}

\begin{table}
  \centering
  \begin{tabular}{|l|r|rr|}
    \hline
    Model & Features & CV AUC & Pub. AUC\\
    \hline
    \hline
    LR & FFT & 0.9337 & 0.6784\\
    \hline
    KNN & FFT & 0.8008 & 0.6759\\
    KNN & FFT+CORW & 0.7994 & 0.7040\\
    \textbf{KNN} & \textbf{PCA+CORW} & \textbf{0.8104} & \textbf{0.7288}\\
    \textbf{KNN} & \textbf{ICA+CORW} & \textbf{0.8103} & \textbf{0.6840}\\
    \hline
    ANN2 & FFT+CORW & 0.9072 & 0.7489\\
    ANN2 & PCA+CORW & 0.9082 & 0.7815\\
    \textbf{ANN2p} & \textbf{PCA+CORW} & \textbf{0.9175} & \textbf{0.7895}\\
    \textbf{ANN2} & \textbf{ICA+CORW} & \textbf{0.9104} & \textbf{0.7772}\\
    ANN3 & PCA+CORW & 0.9188 & 0.7690\\
    ANN4 & PCA+CORW & 0.9268 & 0.7772\\
    \textbf{ANN5} & \textbf{PCA+CORW} & \textbf{0.9283} & \textbf{0.7937}\\
    ANN6 & PCA+CORW & 0.9291 & 0.7722\\
    \hline
    \textbf{KNN} & \textbf{CORG} & \textbf{0.7097} & \textbf{0.6552}\\
    \textbf{KNN} & \textbf{COVRED} & \textbf{0.6900} & \textbf{0.6901}\\
    \hline
    UNIFORM & ENSEMBLE & --- & 0.8048\\
    BMC & ENSEMBLE & 0.9271 & 0.8249\\
    \hline
  \end{tabular}
  \caption{Cross-validation and public AUC results for the most important systems tested during the competition.\label{tab:val}}
\end{table}            

Table~\ref{tab:val} shows that logistic regression model was the worst in
terms of generalization abiltiy. The application of PCA allows to
improve the public AUC results by \textbf{0.025} for KNN model and
\textbf{0.033} for ANN model. ICA obtains improvements similar to PCA.
The advantage of deep ANNs (with more than two hidden layers) is not
clear, but the best public AUC was obtained by an ANN with five hidden
layers, improving by \textbf{0.004} the public AUC of the best ANN2.
Global features as CORG and COVRED show strong correlation between
cross-validation AUC and public test AUC. Finally, the use of BMC
ensemble method allow to improve the public AUC in \textbf{0.031} points
compared to the ANN5 model.

Finally, the private AUC for the two selected models is shown at
Table~\ref{tab:private}.

\begin{table}
  \centering
  \begin{tabular}{|l|r|r|}
    \hline
    Model & FEATURES & Priv. AUC\\
    \hline
    \hline
    ANN5 & PCA+CORW & 0.7644\\
    BMC & ENSEMBLE & 0.7935\\
    \hline
  \end{tabular}
  \caption{Private AUC for the two selected submissions.\label{tab:private}}
\end{table}

The private AUC of the BMC ensemble is the second best if we compare it
to all our submissions, however, the difference with the best private
AUC is not significant, the best one has a private AUC of
\textbf{0.7958}.

The experimentation has been performed in a cluster of \textbf{three}
computers with same hardware configuration:

\begin{itemize}
\item
  Dell PowerEdge 210 II server rack.
\item
  Intel Xeon E3-1220 v2 at 3.10GHz with 16GB RAM (4 cores).
\item
  2.6TB of NFS storage.
\end{itemize}

\section{Simple features and methods}\label{simple-features-and-methods}

The best single features plus model was ANN5, an artificial neural
network with 5 hidden layers using ReLUs~\cite{2011:glorot:aistats}
as activation function, with 128 neurons in every layer and logistic
activation function as output. The input features were FFT+CORW as
described above. This system achieves a \textbf{0.7644} AUC in the
private test.

\bibliography{refs}
\bibliographystyle{plain}

\end{document}
